\documentclass[14pt]{extarticle}
\usepackage[a4paper, total={18cm, 27cm}]{geometry}
\usepackage[utf8]{inputenc}
\usepackage{amsmath}
\usepackage[oznake]{omrezja}

\tikzstyle{pot}=[very thick, double]
\tikzstyle{dosezeno}=[fill=lightgray]
\tikzstyle{pogojno}=[fill=pink]
\tikzstyle{vozlisce pretoka}=[vozlisce, minimum size=8mm]
\tikzstyle{velikost}=[scale=0.9]

\NewEnviron{pretok@grafA}{
    \vozlisce*{s}{-8, 5}
    \vozlisce{a}{ 0, 5}
    \vozlisce{b}{-4, 2}
    \vozlisce{c}{-8,-1}
    \vozlisce{d}{ 0,-1}
    \vozlisce{e}{ 4, 2}
    \vozlisce{f}{ 0,-5}
    \vozlisce{g}{ 8, 5}
    \vozlisce*{t}{ 8,-1}

    \povezava{s}{a}{6}{above}
    \povezava{s}{b}{3}[sloped]{above}
    \povezava{s}{c}{2}{left}
    \povezava{a}{b}{2}[sloped]{above}
    \povezava{a}{e}{9}[sloped]{above}
    \povezava{b}{c}{7}[sloped]{below}
    \povezava{c}{d}{5}{below}
    \povezava{c}{f}{1}[sloped]{below}
    \povezava{d}{a}{10}{left}
    \povezava{d}{b}{10}[sloped]{below}
    \povezava{d}{e}{3}[sloped]{below}
    \povezava{d}{t}{2}{below}
    \povezava{e}{g}{2}[sloped]{above}
    \povezava{e}{t}{5}[sloped]{below}
    \povezava{f}{d}{8}{left}
    \povezava{f}{t}{4}[sloped]{below}
    \povezava{g}{a}{3}{above}
    \povezava{g}{t}{4}{right}
}

\NewEnviron{pretok@grafB}{
    \vozlisce*{s}{-10, 0}
    \vozlisce{a}{ -6, 4.5}
    \vozlisce{b}{ -6, 1.5}
    \vozlisce{c}{ -6,-1.5}
    \vozlisce{d}{ -6,-4.5}
    \vozlisce{e}{ -2, 4.5}
    \vozlisce{f}{ -2, 1.5}
    \vozlisce{g}{ -2,-1.5}
    \vozlisce{h}{ -2,-4.5}
    \vozlisce{i}{  2, 4.5}
    \vozlisce{j}{  2, 1.5}
    \vozlisce{k}{  2,-1.5}
    \vozlisce{l}{  2,-4.5}
    \vozlisce{m}{  6, 4.5}
    \vozlisce{n}{  6, 1.5}
    \vozlisce{o}{  6,-1.5}
    \vozlisce{p}{  6,-4.5}
    \vozlisce*{t}{ 10, 0}

    \povezava{s}{a}{7}[sloped]{above}[bend left=15]
    \povezava{s}{b}{6}[sloped]{above}
    \povezava{s}{c}{7}[sloped]{below}
    \povezava{s}{d}{8}[sloped]{below}[bend right=15]
    \povezava{a}{e}{5}{above}
    \povezava{b}{a}{6}{right}
    \povezava{b}{f}{3}{below}
    \povezava{c}{b}{3}{left}
    \povezava{c}{g}{7}{above}
    \povezava{d}{c}{6}{right}
    \povezava{d}{h}{7}{below}
    \povezava{e}{f}{3}{left}
    \povezava{e}{i}{8}{above}
    \povezava{f}{g}{4}{right}
    \povezava{f}{j}{4}{above}
    \povezava{g}{h}{1}{left}
    \povezava{g}{k}{2}{below}
    \povezava{h}{l}{5}{below}
    \povezava{i}{m}{1}{above}
    \povezava{j}{i}{1}{right}
    \povezava{j}{n}{9}{below}
    \povezava{k}{j}{10}{left}
    \povezava{k}{o}{4}{above}
    \povezava{l}{k}{2}{right}
    \povezava{l}{p}{1}{below}
    \povezava{m}{n}{5}{left}
    \povezava{m}{t}{12}[sloped]{above}[bend left=15]
    \povezava{n}{o}{6}{right}
    \povezava{n}{t}{4}[sloped]{above}
    \povezava{o}{p}{6}{left}
    \povezava{o}{t}{8}[sloped]{below}
    \povezava{p}{t}{8}[sloped]{below}[bend right=15]
}

\NewEnviron{pretok@grafC}{
    \vozlisce*{s}{-6, 0}
    \vozlisce{a}{-2, 3}
    \vozlisce{b}{-2,-3}
    \vozlisce{c}{ 2, 0}
    \vozlisce{d}{ 6, 3}
    \vozlisce*{t}{ 6,-3}

    \povezava{s}{a}{\alpha}[sloped]{above}
    \povezava{s}{b}{3}[sloped]{below}
    \povezava{a}{c}{3}[sloped]{below}
    \povezava{a}{d}{4}{above}
    \povezava{b}{c}{2}[sloped]{below}
    \povezava{b}{t}{4}{below}
    \povezava{c}{t}{2}[sloped]{above}
    \povezava{d}{t}{3}{right}
}

\NewEnviron{pretok@grafD}{
    \vozlisce*{s}{-10,0}
    \vozlisce{a}{-6, 6}
    \vozlisce{b}{-6, 0}
    \vozlisce{c}{-6,-6}
    \vozlisce{d}{-2, 3}
    \vozlisce{e}{-2,-3}
    \vozlisce{f}{ 2,-3}
    \vozlisce{g}{ 6, 6}
    \vozlisce{h}{ 6, 3}
    \vozlisce{i}{ 6, 0}
    \vozlisce{j}{ 6,-6}
    \vozlisce*{t}{10, 0}

    \povezava{s}{a}{8}[sloped]{above}
    \povezava{s}{b}{12}{above}
    \povezava{s}{c}{20}[sloped]{below}
    \povezava{a}{g}{5}{above}
    \povezava{b}{a}{5}{left}
    \povezava{b}{c}{5}{left}
    \povezava{b}{d}{7}[sloped]{below}
    \povezava{b}{e}{3}[sloped]{below}
    \povezava{c}{e}{10}[sloped]{below}
    \povezava{c}{j}{6}{below}
    \povezava{g}{t}{10}[sloped]{above}
    \povezava{d}{a}{2}[sloped]{above}
    \povezava{d}{h}{3}{above}
    \povezava{d}{f}{8}[sloped]{below}
    \povezava{e}{f}{8}{below}
    \povezava{h}{g}{3}{left}
    \povezava{h}{t}{4}[sloped]{below}
    \povezava{f}{i}{10}[sloped]{above}
    \povezava{f}{j}{2}[sloped]{below}
    \povezava{i}{h}{7}{left}
    \povezava{i}{t}{3}[sloped]{below}
    \povezava{j}{i}{4}{left}
    \povezava{j}{t}{8}[sloped]{below}
}

\NewEnviron{pretok@grafE}{
    \vozlisce*{s}{-8, 0}
    \vozlisce{a}{-4, 4}
    \vozlisce{b}{-4, 0}
    \vozlisce{c}{-4,-4}
    \vozlisce{d}{ 0, 4}
    \vozlisce{e}{ 0, 0}
    \vozlisce{f}{ 4, 4}
    \vozlisce{g}{ 4, 0}
    \vozlisce{h}{ 4,-4}
    \vozlisce{i}{ 8, 4}
    \vozlisce{j}{ 8,-4}
    \vozlisce*{t}{ 8, 0}

    \povezava{s}{a}{5}[sloped]{above}
    \povezava{s}{b}{\infty}{above}
    \povezava{s}{c}{13}[sloped]{below}
    \povezava{a}{d}{\infty}{above}
    \povezava{a}{e}{\infty}[sloped]{below}
    \povezava{b}{a}{\infty}{left}
    \povezava{b}{c}{21}{left}
    \povezava{c}{e}{9}[sloped]{above}
    \povezava{c}{h}{\infty}{below}
    \povezava{d}{e}{\infty}{right}
    \povezava{d}{f}{5}{above}
    \povezava{e}{g}{23}{below}
    \povezava{f}{g}{3}{left}
    \povezava{h}{g}{\infty}{left}
    \povezava{g}{i}{\infty}[sloped]{below}
    \povezava{g}{j}{16}[sloped]{above}
    \povezava{i}{f}{11}{above}
    \povezava{i}{t}{\infty}{right}
    \povezava{j}{h}{\infty}{below}
    \povezava{j}{t}{\infty}{right}
}


\title{Optimizacijske metode -- vaje}
\author{Problem pretoka}
\date{24.4.2020}

\begin{document}

\maketitle

\section*{Naloga 1}

Poišči maksimalni pretok in minimalni prerez na sledečem grafu.

\begin{pretok}{grafA}
\end{pretok}

\clearpage

\subsection*{Rešitev}

Poiščemo disjunktne povečujoče poti.

\begin{pretok}{grafA}
    \nicle
    \kolicina*{s}{a}[6]{0}[red]
    \kolicina*{a}{e}[9]{0}[red]
    \kolicina*{e}{t}[5]{0}[red]
    \kolicina*{s}{b}[3]{0}[green]
    \kolicina*{b}{c}[7]{0}[green]
    \kolicina*{c}{d}[5]{0}[green]
    \kolicina*{d}{t}[2]{0}[green]
    \kolicina*{s}{c}[2]{0}[blue]
    \kolicina*{c}{f}[1]{0}[blue]
    \kolicina*{f}{t}[4]{0}[blue]
\end{pretok}

\begin{pretok}{grafA}
    \nicle
    \kolicina*{s}{a}[1]{5}[red]
    \kolicina*{a}{e}[4]{5}[red]
    \kolicina*{e}{g}[2]{0}[red]
    \kolicina*{g}{t}[4]{0}[red]
    \kolicina{e}{t}[0]{5}[zasicena]
    \kolicina{s}{b}[1]{2}
    \kolicina{b}{c}[5]{2}
    \kolicina{c}{d}[3]{2}
    \kolicina{d}{t}[0]{2}[zasicena]
    \kolicina{s}{c}[1]{1}
    \kolicina{c}{f}[0]{1}[zasicena]
    \kolicina{f}{t}[3]{1}

    \dosezeno{s}
    \dosezeno{b}
    \dosezeno{c}
    \dosezeno{d}
    \dosezeno{e}
    \dosezeno{a}
\end{pretok}

\begin{pretok}{grafA}
    \nicle
    \kolicina*{s}{c}[1]{1}[red]
    \kolicina*{c}{d}[3]{2}[red]
    \kolicina*{d}{e}[3]{2}[red]
    \kolicina*{e}{g}[1]{1}[red]
    \kolicina*{g}{t}[3]{1}[red]
    \kolicina{s}{a}[0]{6}[zasicena]
    \kolicina{a}{e}[3]{6}
    \kolicina{e}{t}[0]{5}[zasicena]
    \kolicina{s}{b}[1]{2}
    \kolicina{b}{c}[5]{2}
    \kolicina{d}{t}[0]{2}[zasicena]
    \kolicina{c}{f}[0]{1}[zasicena]
    \kolicina{f}{t}[3]{1}

    \dosezeno{s}
    \dosezeno{b}
    \dosezeno{c}
\end{pretok}

\begin{pretok}{grafA}
    \nicle
    \kolicina{s}{c}[0]{2}[zasicena]
    \kolicina{c}{d}[2]{3}
    \kolicina{d}{e}[2]{3}
    \kolicina{e}{g}[0]{2}[prerez]
    \kolicina{g}{t}[2]{2}
    \kolicina{s}{a}[0]{6}[zasicena]
    \kolicina{a}{e}[3]{6}
    \kolicina{e}{t}[0]{5}[prerez]
    \kolicina{s}{b}[1]{2}
    \kolicina{b}{c}[5]{2}
    \kolicina{d}{t}[0]{2}[prerez]
    \kolicina{c}{f}[0]{1}[prerez]
    \kolicina{f}{t}[3]{1}

    \nasprotna{g}{a}
    \nasprotna{f}{d}

    \dosezeno{s}
    \dosezeno{b}
    \dosezeno{c}
    \dosezeno{d}
    \dosezeno{e}
    \dosezeno{a}
\end{pretok}

Maksimalni pretok: $6+2+2 = 1+2+5+2 = 10$,
minimalni prerez $eg, et, dt, cf$, kapaciteta $2+5+2+1 = 10$.

\clearpage

\section*{Naloga 2}

Poišči maksimalni pretok in minimalni prerez na sledečem grafu.

\begin{pretok}{grafB}
\end{pretok}

\subsection*{Rešitev}

\begin{pretok}{grafB}
    \nicle
    \kolicina*{s}{a}[7]{0}[red]
    \kolicina*{a}{e}[5]{0}[red]
    \kolicina*{e}{i}[8]{0}[red]
    \kolicina*{i}{m}[1]{0}[red]
    \kolicina*{m}{t}[12]{0}[red]
    \kolicina*{s}{b}[6]{0}[green]
    \kolicina*{b}{f}[3]{0}[green]
    \kolicina*{f}{j}[4]{0}[green]
    \kolicina*{j}{n}[9]{0}[green]
    \kolicina*{n}{t}[4]{0}[green]
    \kolicina*{s}{c}[7]{0}[blue]
    \kolicina*{c}{g}[7]{0}[blue]
    \kolicina*{g}{k}[2]{0}[blue]
    \kolicina*{k}{o}[4]{0}[blue]
    \kolicina*{o}{t}[8]{0}[blue]
    \kolicina*{s}{d}[8]{0}[magenta]
    \kolicina*{d}{h}[7]{0}[magenta]
    \kolicina*{h}{l}[5]{0}[magenta]
    \kolicina*{l}{p}[1]{0}[magenta]
    \kolicina*{p}{t}[8]{0}[magenta]
\end{pretok}

\begin{pretok}{grafB}
    \nicle
    \kolicina*{s}{c}[5]{2}[red]
    \kolicina*{c}{g}[5]{2}[red]
    \kolicina*{g}{h}[1]{0}[red]
    \kolicina*{h}{l}[4]{1}[red]
    \kolicina*{l}{k}[2]{0}[red]
    \kolicina*{k}{o}[2]{2}[red]
    \kolicina*{o}{t}[6]{2}[red]
    \kolicina*{s}{a}[6]{1}[green]
    \kolicina*{a}{e}[4]{1}[green]
    \kolicina*{e}{f}[3]{0}[green]
    \kolicina*{f}{j}[1]{3}[green]
    \kolicina*{j}{n}[6]{3}[green]
    \kolicina*{n}{t}[1]{3}[green]
    \kolicina{e}{i}[7]{1}
    \kolicina{i}{m}[0]{1}[zasicena]
    \kolicina{m}{t}[11]{1}
    \kolicina{s}{b}[3]{3}
    \kolicina{b}{f}[0]{3}[zasicena]
    \kolicina{g}{k}[0]{2}[zasicena]
    \kolicina{s}{d}[7]{1}
    \kolicina{d}{h}[6]{1}
    \kolicina{l}{p}[0]{1}[zasicena]
    \kolicina{p}{t}[7]{1}

    \dosezeno{s}
    \dosezeno{b}
    \dosezeno{d}
    \dosezeno{a}
    \dosezeno{c}
    \dosezeno{h}
\end{pretok}

\begin{pretok}{grafB}
    \nicle
    \kolicina*{s}{d}[7]{1}[red]
    \kolicina*{d}{h}[6]{1}[red]
    \kolicina*{h}{l}[3]{2}[red]
    \kolicina*{l}{k}[1]{1}[red]
    \kolicina*{k}{o}[1]{3}[red]
    \kolicina*{o}{t}[5]{3}[red]
    \kolicina{s}{c}[4]{3}
    \kolicina{c}{g}[4]{3}
    \kolicina{g}{h}[0]{1}[zasicena]
    \kolicina{s}{a}[5]{2}
    \kolicina{a}{e}[3]{2}
    \kolicina{e}{f}[2]{1}
    \kolicina{f}{j}[0]{4}[zasicena]
    \kolicina{j}{n}[5]{4}
    \kolicina{n}{t}[0]{4}[zasicena]
    \kolicina{e}{i}[7]{1}
    \kolicina{i}{m}[0]{1}[zasicena]
    \kolicina{m}{t}[11]{1}
    \kolicina{s}{b}[3]{3}
    \kolicina{b}{f}[0]{3}[zasicena]
    \kolicina{g}{k}[0]{2}[zasicena]
    \kolicina{l}{p}[0]{1}[zasicena]
    \kolicina{p}{t}[7]{1}

    \dosezeno{s}
    \dosezeno{a}
    \dosezeno{b}
    \dosezeno{c}
    \dosezeno{e}
    \dosezeno{g}
    \dosezeno{f}
    \dosezeno{i}
\end{pretok}

\begin{pretok}{grafB}
    \nicle
    \kolicina{s}{d}[6]{2}
    \kolicina{d}{h}[5]{2}
    \kolicina{h}{l}[2]{3}
    \kolicina{l}{k}[0]{2}[prerez]
    \kolicina{k}{o}[0]{4}[zasicena]
    \kolicina{o}{t}[4]{4}
    \kolicina{s}{c}[4]{3}
    \kolicina{c}{g}[4]{3}
    \kolicina{g}{h}[0]{1}[zasicena]
    \kolicina{s}{a}[5]{2}
    \kolicina{a}{e}[3]{2}
    \kolicina{e}{f}[2]{1}
    \kolicina{f}{j}[0]{4}[prerez]
    \kolicina{j}{n}[5]{4}
    \kolicina{n}{t}[0]{4}[zasicena]
    \kolicina{e}{i}[7]{1}
    \kolicina{i}{m}[0]{1}[prerez]
    \kolicina{m}{t}[11]{1}
    \kolicina{s}{b}[3]{3}
    \kolicina{b}{f}[0]{3}[zasicena]
    \kolicina{g}{k}[0]{2}[prerez]
    \kolicina{l}{p}[0]{1}[prerez]
    \kolicina{p}{t}[7]{1}

    \nasprotna{j}{i}

    \dosezeno{s}
    \dosezeno{a}
    \dosezeno{b}
    \dosezeno{c}
    \dosezeno{e}
    \dosezeno{g}
    \dosezeno{f}
    \dosezeno{i}
    \dosezeno{d}
    \dosezeno{h}
    \dosezeno{l}
\end{pretok}

Maksimalni pretok $2+3+3+2 = 1+4+4+1 = 10$, minimalni prerez $fj, fk, im, \ell k,  \ell p$, kapaciteta $4+2+1+2+1 = 10$.

\clearpage

\section*{Naloga 3}

Poišči maksimalni pretok in minimalni prerez na sledečem grafu v odvisnosti od parametra $\alpha$.

\begin{pretok}{grafC}
\end{pretok}

\subsection*{Rešitev}

Najprej rešimo nalogo pri $\alpha = 0$.

\begin{pretok}{grafC}
    \nicle
    \kolicina*{s}{b}[3]{0}[red]
    \kolicina*{b}{t}[4]{0}[red]
    \kolicina*{s}{a}[\alpha]{0}[green]
    \kolicina*{a}{d}[4]{0}[green]
    \kolicina*{d}{t}[3]{0}[green]
\end{pretok}

\clearpage

Če je $0 \le \alpha < 3$, potem imamo maksimalni pretok $\alpha + 3$.

\begin{pretok}{grafC}
    \nicle
    \kolicina{s}{b}[0]{3}[prerez]
    \kolicina{b}{t}[1]{3}
    \kolicina{s}{a}[0]{\alpha}[prerez]
    \kolicina{a}{d}[4-\alpha]{\alpha}
    \kolicina{d}{t}[3-\alpha]{\alpha}

    \dosezeno{s}
\end{pretok}

Minimalni prerez: $sa, sb$, kapaciteta $\alpha + 3$.

\medskip
Če je $\alpha \ge 3$:

\begin{pretok}{grafC}
    \nicle
    \kolicina*{s}{a}[\alpha-3]{3}[red]
    \kolicina*{a}{c}[3]{0}[red]
    \kolicina*{c}{t}[2]{0}[red]
    \kolicina{s}{b}[0]{3}[zasicena]
    \kolicina{b}{t}[1]{3}
    \kolicina{a}{d}[1]{3}
    \kolicina{d}{t}[0]{3}[zasicena]
\end{pretok}

Če je $3 \le \alpha < 5$, potem imamo maksimalni pretok $\alpha + 3$.

\begin{pretok}{grafC}
    \nicle
    \kolicina{s}{a}[0]{\alpha}[prerez]
    \kolicina{a}{c}[6-\alpha]{\alpha-3}
    \kolicina{c}{t}[5-\alpha]{\alpha-3}
    \kolicina{s}{b}[0]{3}[prerez]
    \kolicina{b}{t}[1]{3}
    \kolicina{a}{d}[1]{3}
    \kolicina{d}{t}[0]{3}[zasicena]

    \dosezeno{s}
\end{pretok}

Minimalni prerez: $sa, sb$, kapaciteta $\alpha + 3$.

\clearpage
Če je $\alpha \ge 5$:

\begin{pretok}{grafC}
    \nicle
    \kolicina{s}{a}[\alpha-5]{5}
    \kolicina{a}{c}[1]{2}
    \kolicina{c}{t}[0]{2}[prerez]
    \kolicina{s}{b}[0]{3}[prerez]
    \kolicina{b}{t}[1]{3}
    \kolicina{a}{d}[1]{3}
    \kolicina{d}{t}[0]{3}[prerez]

    \nasprotna{b}{c}

    \dosezeno{s}
    \dosezeno{a}
    \dosezeno{c}
    \dosezeno{d}
\end{pretok}

Maksimalni pretok $5+3 = 3+2+3 = 8$, minimalni prerez $sb, ct, dt$, kapaciteta $3+2+3 = 8$.

\clearpage

\section*{Naloga 4}

V sledečem grafu lahko kapaciteto ene povezave povečamo za $1$. Katera naj bo ta povezava, da bomo povečali maksimalni pretok?

\begin{pretok}{grafD}
\end{pretok}

\clearpage

\subsection*{Rešitev}

Najprej poiščemo maksimalni pretok in minimalni prerez.

\begin{pretok}{grafD}
    \nicle
    \kolicina*{s}{c}[20]{0}[red]
    \kolicina*{c}{j}[6]{0}[red]
    \kolicina*{j}{t}[8]{0}[red]
    \kolicina*{s}{a}[8]{0}[green]
    \kolicina*{a}{g}[5]{0}[green]
    \kolicina*{g}{t}[10]{0}[green]
    \kolicina*{s}{b}[12]{0}[blue]
    \kolicina*{b}{d}[7]{0}[blue]
    \kolicina*{d}{f}[8]{0}[blue]
    \kolicina*{f}{i}[10]{0}[blue]
    \kolicina*{i}{h}[7]{0}[blue]
    \kolicina*{h}{t}[4]{0}[blue]
\end{pretok}

\begin{pretok}{grafD}
    \nicle
    \kolicina*{s}{b}[8]{4}[red]
    \kolicina*{b}{d}[3]{4}[red]
    \kolicina*{d}{f}[4]{4}[red]
    \kolicina*{f}{i}[6]{4}[red]
    \kolicina*{i}{t}[3]{0}[red]
    \kolicina*{s}{c}[14]{6}[green]
    \kolicina*{c}{e}[10]{0}[green]
    \kolicina*{e}{f}[8]{0}[green]
    \kolicina*{f}{j}[2]{0}[green]
    \kolicina*{j}{t}[2]{6}[green]
    \kolicina{c}{j}[0]{6}[zasicena]
    \kolicina{s}{a}[3]{5}
    \kolicina{a}{g}[0]{5}[zasicena]
    \kolicina{g}{t}[5]{5}
    \kolicina{i}{h}[3]{4}
    \kolicina{h}{t}[0]{4}[zasicena]

    \dosezeno{s}
    \dosezeno{a}
\end{pretok}

\begin{pretok}{grafD}
    \nicle
    \kolicina*{s}{b}[5]{7}[red]
    \kolicina*{b}{e}[3]{0}[red]
    \kolicina*{e}{f}[6]{2}[red]
    \kolicina*{f}{i}[3]{7}[red]
    \kolicina*{i}{h}[3]{4}[red]
    \kolicina*{h}{g}[3]{0}[red]
    \kolicina*{g}{t}[5]{5}[red]
    \kolicina{b}{d}[0]{7}[zasicena]
    \kolicina{d}{f}[1]{7}
    \kolicina{i}{t}[0]{3}[zasicena]
    \kolicina{s}{c}[12]{8}
    \kolicina{c}{e}[8]{2}
    \kolicina{f}{j}[0]{2}[zasicena]
    \kolicina{j}{t}[0]{8}[zasicena]
    \kolicina{c}{j}[0]{6}[zasicena]
    \kolicina{s}{a}[3]{5}
    \kolicina{a}{g}[0]{5}[zasicena]
    \kolicina{h}{t}[0]{4}[zasicena]

    \dosezeno{s}
    \dosezeno{a}
    \dosezeno{c}
    \dosezeno{e}
\end{pretok}

\begin{pretok}{grafD}
    \nicle
    \kolicina{s}{b}[2]{10}
    \kolicina{b}{e}[0]{3}[zasicena]
    \kolicina{e}{f}[3]{5}
    \kolicina{f}{i}[0]{10}[zasicena]
    \kolicina{i}{h}[0]{7}[zasicena]
    \kolicina{h}{g}[0]{3}[prerez]
    \kolicina{g}{t}[2]{8}
    \kolicina{b}{d}[0]{7}[zasicena]
    \kolicina{d}{f}[1]{7}
    \kolicina{i}{t}[0]{3}[prerez]
    \kolicina{s}{c}[12]{8}
    \kolicina{c}{e}[8]{2}
    \kolicina{f}{j}[0]{2}[prerez]
    \kolicina{j}{t}[0]{8}[zasicena]
    \kolicina{c}{j}[0]{6}[prerez]
    \kolicina{s}{a}[3]{5}
    \kolicina{a}{g}[0]{5}[prerez]
    \kolicina{h}{t}[0]{4}[prerez]

    \nasprotna{j}{i}

    \dosezeno{s}
    \dosezeno{a}
    \dosezeno{c}
    \dosezeno{e}
    \dosezeno{b}
    \dosezeno{f}
    \dosezeno{d}
    \dosezeno{h}
    \dosezeno{i}
\end{pretok}

Maksimalni pretok $5+10+8 = 8+4+3+8 = 23$, \\
minimalni prerez $ag, cj, fj, hg, ht, it$,
kapaciteta $5+6+2+3+4+3 = 23$.

\clearpage

Da povečamo kapaciteto omrežja,
lahko povečamo kapaciteto katerikoli povezavi iz minimalnega prereza.
Povečajmo kapaciteto povezavi $it$ za $1$:

\begin{pretok}{grafD}
    \nicle
    \kolicina*{s}{c}[12]{8}[magenta]
    \kolicina*{c}{e}[8]{2}[magenta]
    \kolicina*{e}{f}[3]{5}[magenta]
    \kolicina*{d}{f}[1]{7}[magenta]
    \kolicina*{d}{h}[3]{0}[magenta]
    \kolicina*{i}{h}[0]{7}[zasicena,magenta]
    \kolicina*{i}{t}[1]{3}[magenta]
    \kolicina{s}{b}[2]{10}
    \kolicina{b}{e}[0]{3}[zasicena]
    \kolicina{f}{i}[0]{10}[zasicena]
    \kolicina{h}{g}[0]{3}[prerez]
    \kolicina{g}{t}[2]{8}
    \kolicina{b}{d}[0]{7}[zasicena]
    \kolicina{f}{j}[0]{2}[prerez]
    \kolicina{j}{t}[0]{8}[zasicena]
    \kolicina{c}{j}[0]{6}[prerez]
    \kolicina{s}{a}[3]{5}
    \kolicina{a}{g}[0]{5}[prerez]
    \kolicina{h}{t}[0]{4}[prerez]

    \nasprotna{j}{i}

    \dosezeno{s}
    \dosezeno{a}
    \dosezeno{c}
    \dosezeno{e}
    \dosezeno{b}
    \dosezeno{f}
    \dosezeno{d}
    \dosezeno{h}
    \dosezeno{i}
    \dosezeno{t}[pogojno]
\end{pretok}

\begin{pretok}{grafD}
    \nicle
    \kolicina{s}{c}[11]{9}
    \kolicina{c}{e}[7]{3}
    \kolicina{e}{f}[2]{5}
    \kolicina{d}{f}[2]{6}
    \kolicina{d}{h}[2]{1}
    \kolicina{i}{h}[1]{6}
    \kolicina{i}{t}[0]{4}[prerez]
    \kolicina{s}{b}[2]{10}
    \kolicina{b}{e}[0]{3}[zasicena]
    \kolicina{f}{i}[0]{10}[zasicena]
    \kolicina{h}{g}[0]{3}[prerez]
    \kolicina{g}{t}[2]{8}
    \kolicina{b}{d}[0]{7}[zasicena]
    \kolicina{f}{j}[0]{2}[prerez]
    \kolicina{j}{t}[0]{8}[zasicena]
    \kolicina{c}{j}[0]{6}[prerez]
    \kolicina{s}{a}[3]{5}
    \kolicina{a}{g}[0]{5}[prerez]
    \kolicina{h}{t}[0]{4}[prerez]

    \nasprotna{j}{i}

    \dosezeno{s}
    \dosezeno{a}
    \dosezeno{c}
    \dosezeno{e}
    \dosezeno{b}
    \dosezeno{f}
    \dosezeno{d}
    \dosezeno{h}
    \dosezeno{i}
\end{pretok}

\clearpage

\section*{Naloga 5}

Poišči maksimalen pretok v sledečem grafu, če ta obstaja!

\begin{pretok}{grafE}
\end{pretok}

\end{document}
