\documentclass[14pt]{extarticle}
\usepackage[a4paper, total={18cm, 26cm}]{geometry}
\usepackage[utf8]{inputenc}
\usepackage[oznake]{omrezja}

\tikzstyle{drevo}=[very thick, double, red]
\tikzstyle{zasicena}=[very thick, double]
\tikzstyle{velikost}=[scale=1]

\NewEnviron{razvoz@grafA}{
    \vozlisce{a}{-7, 2}[-4]{above}
    \vozlisce{b}{-7,-2}[-2]{below}
    \vozlisce{c}{-2, 2}{above}
    \vozlisce{d}{-2,-2}[-5]{below}
    \vozlisce{e}{1, 0}{above}
    \vozlisce{f}{4, 2}{above}
    \vozlisce{g}{4,-2}[3]{below}
    \vozlisce{h}{7, 0}[8]{right}

    \povezava{a}{c}{6}{above}
    \povezava{a}{d}{2}[sloped]{above}
    \povezava{b}{a}{1}{left}
    \povezava{b}{d}{2}{below}
    \povezava{c}{b}{1}[sloped]{above}
    \povezava{c}{d}{1}{left}
    \povezava{c}{e}{2}[sloped]{below}
    \povezava{d}{e}{8}[sloped]{above}
    \povezava{d}{g}{2}{below}
    \povezava{e}{f}{1}[sloped]{above}
    \povezava{e}{g}{1}[sloped]{below}
    \povezava{e}{h}{2}{below}
    \povezava{f}{c}{2}{above}
    \povezava{f}{h}{7}[sloped]{above}
    \povezava{g}{f}{1}{right}
    \povezava{g}{h}{8}[sloped]{below}
}

\NewEnviron{razvoz@grafB}{
    \vozlisce{a}{-4, 3}[-10]{above}
    \vozlisce{b}{-4, 0}[-10]{45}
    \vozlisce{c}{-4,-3}[-5]{below}
    \vozlisce{d}{ 0, 3}{above}
    \vozlisce{e}{ 0, 0}{45}
    \vozlisce{f}{ 0,-3}{below}
    \vozlisce{g}{ 4, 3}{above}
    \vozlisce{h}{ 4, 0}{135}
    \vozlisce{i}{ 4,-3}[10]{below}
    \vozlisce{j}{ 8, 3}[10]{above}
    \vozlisce{k}{ 8, 0}{right}
    \vozlisce{l}{ 8,-3}{below}

    \povezava{a}{d}{2}{above}
    \povezava{b}{a}{3}{left}
    \povezava{b}{c}{12}{left}
    \povezava{b}{e}{15}{below}
    \povezava{c}{f}{5}{below}
    \povezava{d}{g}{3}{above}
    \povezava{e}{d}{8}{left}
    \povezava{e}{f}{3}{left}
    \povezava{f}{i}{1}{below}
    \povezava{g}{h}{2}{right}
    \povezava{g}{j}{15}{above}
    \povezava{h}{e}{3}{below}
    \povezava{h}{i}{10}{right}
    \povezava{h}{k}{10}{below}
    \povezava{i}{l}{4}{below}
    \povezava{k}{j}{2}{right}
    \povezava{l}{k}{-3}{right}
}

\NewEnviron{razvoz@grafBs}{
    \begin{razvoz@grafB}
    \end{razvoz@grafB}

    \vozlisce{s}{-8, 0}[5]{left}
    \povezava{a}{s}{0}[sloped]{above}
    \povezava{b}{s}{0}{below}
    \povezava{c}{s}{0}[sloped]{below}
}

\NewEnviron{razvoz@grafC}{
    \vozlisce{a}{-8, 5}[1]{left}
    \vozlisce{b}{ 8, 5}[5]{right}
    \vozlisce{c}{ 8,-5}[-4]{right}
    \vozlisce{d}{-8,-5}[-2]{left}
    \vozlisce{e}{-3, 3}[10]{above}
    \vozlisce{f}{ 3, 3}[-7]{above}
    \vozlisce{g}{ 0,-2}[-3]{below}

    \povezava{a}{b}{3}{above}
    \povezava{c}{b}{9}{right}
    \povezava{c}{d}{4}{below}
    \povezava{d}{a}{1}{left}
    \povezava{e}{a}{2}[sloped]{above}
    \povezava{e}{d}{11}[sloped]{above}
    \povezava{f}{b}{4}[sloped]{above}
    \povezava{f}{e}{8}{above}
    \povezava{g}{c}{2}[sloped]{below}
    \povezava{g}{d}{17}[sloped]{below}
    \povezava{g}{e}{1}[sloped]{above}
    \povezava{g}{f}{9}[sloped]{above}
}

\NewEnviron{razvoz@grafCs}{
    \cena{a}{b}{0}
    \cena{c}{b}{0}
    \cena{c}{d}{0}
    \cena{d}{a}{0}
    \cena{e}{a}{0}
    \cena{e}{d}{0}
    \cena{f}{b}{0}
    \cena{f}{e}{0}
    \cena{g}{c}{0}
    \cena{g}{d}{0}
    \cena{g}{e}{0}
    \cena{g}{f}{0}

    \begin{razvoz@grafC}
    \end{razvoz@grafC}

    \povezava{c}{g}{1}[sloped]{above}[bend right=15]
    \povezava{d}{g}{1}[sloped]{above}[bend left=15]
    \povezava{f}{g}{1}[sloped]{below}[bend left=15]
    \povezava{g}{a}{1}[sloped]{below}
    \povezava{g}{b}{1}[sloped]{below}[bend right=20]
}

\NewEnviron{razvoz@grafD}{
    \vozlisce{a}{-4, 3}[-5]{above}
    \vozlisce{b}{-4, 0}[-5]{left}
    \vozlisce{c}{-4,-3}{below}
    \vozlisce{d}{ 0, 3}{above}
    \vozlisce{e}{ 0, 0}{45}
    \vozlisce{f}{ 0,-3}{below}
    \vozlisce{g}{ 4, 3}{above}
    \vozlisce{h}{ 4, 0}{135}
    \vozlisce{i}{ 4,-3}{below}
    \vozlisce{j}{ 8, 3}[7]{above}
    \vozlisce{k}{ 8, 0}[3]{right}
    \vozlisce{l}{ 8,-3}{below}

    \povezava{a}{d}{5}{above}
    \povezava{b}{a}{10}{left}
    \povezava{b}{c}{4}{left}
    \povezava{b}{e}{10}{below}
    \povezava{c}{f}{4}{below}
    \povezava{d}{e}{3}{left}
    \povezava{e}{f}{5}{left}
    \povezava{e}{h}{a}{below}
    \povezava{f}{i}{2}{below}
    \povezava{g}{d}{1}{above}
    \povezava{g}{j}{15}{above}
    \povezava{h}{f}{3}[sloped]{below}
    \povezava{h}{g}{6}{right}
    \povezava{h}{k}{10}{below}
    \povezava{i}{h}{3}{right}
    \povezava{i}{l}{3}{below}
    \povezava{k}{j}{2}{right}
    \povezava{l}{k}{1}{right}
}


\title{Optimizacijske metode -- vaje}
\author{Problem razvoza}
\date{27.3.2020}

\begin{document}

\maketitle

\section*{Naloga 1}

Reši problem razvoza na grafu s simpleksno metodo za omrežje.

\begin{razvoz}{grafA}
\end{razvoz}

\subsection*{Rešitev}

Najprej poiščemo drevesno dopustno rešitev.

\begin{razvoz}{grafA}
    \kolicina{b}{a}{2}[izstopi]
    \kolicina{a}{c}{0}
    \kolicina{a}{d}{6}
    \kolicina{d}{e}{8}
    \kolicina{d}{g}{3}
    \kolicina{e}{f}{8}
    \kolicina{f}{h}{8}

    \oznaka{b}{0}
    \oznaka{a}{1}
    \oznaka{c}{7}
    \oznaka{d}{3}
    \oznaka{e}{11}
    \oznaka{g}{5}
    \oznaka{f}{12}
    \oznaka{h}{19}

    \kandidat{b}{d}[vstopi]
    \kandidat{c}{e}
    \kandidat{e}{h}
    \kandidat{g}{f}
    \kandidat{g}{h}
\end{razvoz}

\begin{razvoz}{grafA}
    \kolicina{a}{c}{0}
    \kolicina{a}{d}{4}[izstopi]
    \kolicina{b}{d}{2}
    \kolicina{d}{e}{8}
    \kolicina{d}{g}{3}
    \kolicina{e}{f}{8}
    \kolicina{f}{h}{8}

    \oznaka{b}{1}
    \oznaka{a}{1}
    \oznaka{c}{7}
    \oznaka{d}{3}
    \oznaka{e}{11}
    \oznaka{g}{5}
    \oznaka{f}{12}
    \oznaka{h}{19}

    \kandidat{c}{e}[vstopi]
    \kandidat{e}{h}
    \kandidat{g}{f}
    \kandidat{g}{h}
\end{razvoz}

\begin{razvoz}{grafA}
    \kolicina{a}{c}{4}
    \kolicina{b}{d}{2}[izstopi]
    \kolicina{d}{e}{4}
    \kolicina{d}{g}{3}
    \kolicina{e}{f}{8}
    \kolicina{f}{h}{8}
    \kolicina{c}{e}{4}

    \oznaka{b}{1}
    \oznaka{a}{3}
    \oznaka{c}{9}
    \oznaka{d}{3}
    \oznaka{e}{11}
    \oznaka{g}{5}
    \oznaka{f}{12}
    \oznaka{h}{19}

    \kandidat{b}{a}[vstopi]
    \kandidat{e}{h}
    \kandidat{g}{f}
    \kandidat{g}{h}
\end{razvoz}

\begin{razvoz}{grafA}
    \kolicina{a}{c}{6}
    \kolicina{b}{a}{2}
    \kolicina{d}{e}{2}
    \kolicina{d}{g}{3}
    \kolicina{e}{f}{8}[izstopi]
    \kolicina{f}{h}{8}
    \kolicina{c}{e}{6}

    \oznaka{b}{2}
    \oznaka{a}{3}
    \oznaka{c}{9}
    \oznaka{d}{3}
    \oznaka{e}{11}
    \oznaka{g}{5}
    \oznaka{f}{12}
    \oznaka{h}{19}

    \kandidat{e}{h}[vstopi]
    \kandidat{g}{f}
    \kandidat{g}{h}
\end{razvoz}

\begin{razvoz}{grafA}
    \kolicina{a}{c}{6}
    \kolicina{b}{a}{2}
    \kolicina{d}{e}{2}
    \kolicina{d}{g}{3}
    \kolicina{e}{h}{8}
    \kolicina{f}{h}{0}[izstopi]
    \kolicina{c}{e}{6}

    \oznaka{b}{2}
    \oznaka{a}{3}
    \oznaka{c}{9}
    \oznaka{d}{3}
    \oznaka{e}{11}
    \oznaka{g}{5}
    \oznaka{f}{6}
    \oznaka{h}{13}

    \kandidat{f}{c}[vstopi]
\end{razvoz}

\begin{razvoz}{grafA}
    \kolicina{a}{c}{6}
    \kolicina{b}{a}{2}
    \kolicina{d}{e}{2}[izstopi]
    \kolicina{d}{g}{3}
    \kolicina{e}{h}{8}
    \kolicina{f}{c}{0}
    \kolicina{c}{e}{6}

    \oznaka{b}{2}
    \oznaka{a}{3}
    \oznaka{c}{9}
    \oznaka{d}{3}
    \oznaka{e}{11}
    \oznaka{g}{5}
    \oznaka{f}{7}
    \oznaka{h}{13}

    \kandidat{g}{f}[vstopi]
\end{razvoz}

Optimalna rešitev:

\begin{razvoz}{grafA}
    \kolicina{a}{c}{6}
    \kolicina{b}{a}{2}[izstopi]
    \kolicina{d}{g}{5}
    \kolicina{e}{h}{8}
    \kolicina{f}{c}{2}
    \kolicina{c}{e}{8}
    \kolicina{g}{f}{2}

    \oznaka{b}{2}
    \oznaka{a}{3}
    \oznaka{c}{9}
    \oznaka{d}{4}
    \oznaka{e}{11}
    \oznaka{g}{6}
    \oznaka{f}{7}
    \oznaka{h}{13}

    \kandidat{b}{d}[vstopi]
\end{razvoz}

Cena razvoza: $2 \cdot 1 + 6 \cdot 6 + 8 \cdot 2 + 8 \cdot 2 + 5 \cdot 2 + 2 \cdot 1 + 2 \cdot 2 = 86$

\clearpage

Splošna rešitev ($0 \le x \le 2$):

\begin{razvoz}{grafA}
    \kolicina{a}{c}{6-x}
    \kolicina{b}{a}{2-x}
    \kolicina{d}{g}{5+x}
    \kolicina{e}{h}{8}
    \kolicina{f}{c}{2+x}
    \kolicina{c}{e}{8}
    \kolicina{g}{f}{2+x}
    \kolicina{b}{d}{x}

    \oznaka{b}{2}
    \oznaka{a}{3}
    \oznaka{c}{9}
    \oznaka{d}{4}
    \oznaka{e}{11}
    \oznaka{g}{6}
    \oznaka{f}{7}
    \oznaka{h}{13}
\end{razvoz}

Cena razvoza: $(2-x) \cdot 1 + (6-x) \cdot 6 + 8 \cdot 2 + 8 \cdot 2 + x \cdot 2 + (5+x) \cdot 2 + (2+x) \cdot 1 + (2+x) \cdot 2 = 86$

\clearpage
\section*{Naloga 2}

Reši problem razvoza na grafu s simpleksno metodo za omrežje.
Pazi na skupno povpraševanje in ponudbo!

\begin{razvoz}{grafB}
\end{razvoz}

\subsection*{Rešitev}

Ker je ponudba večja od povpraševanja,
graf dopolnimo z novim vozliščem.

\begin{razvoz}{grafBs}
    \kolicina{a}{d}{10}
    \kolicina{d}{g}{10}
    \kolicina{g}{j}{10}
    \kolicina{b}{e}{10}
    \kolicina{e}{f}{10}
    \kolicina{f}{i}{10}
    \kolicina{c}{s}{5}[izstopi]
    \kolicina{a}{s}{0}
    \kolicina{b}{s}{0}
    \kolicina{g}{h}{0}
    \kolicina{i}{l}{0}
    \kolicina{l}{k}{0}

    \oznaka{s}{0}
    \oznaka{a}{0}
    \oznaka{b}{0}
    \oznaka{c}{0}
    \oznaka{d}{2}
    \oznaka{g}{5}
    \oznaka{j}{20}
    \oznaka{h}{7}
    \oznaka{e}{15}
    \oznaka{f}{18}
    \oznaka{i}{19}
    \oznaka{l}{23}
    \oznaka{k}{20}

    \kandidat{c}{f}[vstopi]
    \kandidat{h}{e}
    \kandidat{h}{i}
    \kandidat{h}{k}
\end{razvoz}

\begin{razvoz}{grafBs}
    \kolicina{a}{d}{10}
    \kolicina{d}{g}{10}
    \kolicina{g}{j}{10}
    \kolicina{b}{e}{5}[izstopi]
    \kolicina{e}{f}{5}
    \kolicina{f}{i}{10}
    \kolicina{a}{s}{0}
    \kolicina{b}{s}{5}
    \kolicina{g}{h}{0}
    \kolicina{i}{l}{0}
    \kolicina{l}{k}{0}
    \kolicina{c}{f}{5}

    \oznaka{s}{0}
    \oznaka{a}{0}
    \oznaka{b}{0}
    \oznaka{c}{13}
    \oznaka{d}{2}
    \oznaka{g}{5}
    \oznaka{j}{20}
    \oznaka{h}{7}
    \oznaka{e}{15}
    \oznaka{f}{18}
    \oznaka{i}{19}
    \oznaka{l}{23}
    \oznaka{k}{20}

    \kandidat{b}{c}[vstopi]
    \kandidat{h}{e}
    \kandidat{h}{i}
    \kandidat{h}{k}
\end{razvoz}

\begin{razvoz}{grafBs}
    \kolicina{a}{d}{10}
    \kolicina{d}{g}{10}
    \kolicina{g}{j}{10}
    \kolicina{b}{c}{5}
    \kolicina{e}{f}{0}
    \kolicina{f}{i}{10}
    \kolicina{a}{s}{0}[izstopi]
    \kolicina{b}{s}{5}
    \kolicina{g}{h}{0}
    \kolicina{i}{l}{0}
    \kolicina{l}{k}{0}
    \kolicina{c}{f}{10}

    \oznaka{s}{0}
    \oznaka{a}{0}
    \oznaka{b}{0}
    \oznaka{c}{12}
    \oznaka{d}{2}
    \oznaka{g}{5}
    \oznaka{j}{20}
    \oznaka{h}{7}
    \oznaka{e}{14}
    \oznaka{f}{17}
    \oznaka{i}{18}
    \oznaka{l}{22}
    \oznaka{k}{19}

    \kandidat{h}{e}[vstopi]
    \kandidat{h}{i}
    \kandidat{h}{k}
\end{razvoz}

\begin{razvoz}{grafBs}
    \kolicina{a}{d}{10}
    \kolicina{d}{g}{10}
    \kolicina{g}{j}{10}
    \kolicina{b}{c}{5}[izstopi]
    \kolicina{e}{f}{0}
    \kolicina{f}{i}{10}
    \kolicina{b}{s}{5}
    \kolicina{g}{h}{0}
    \kolicina{i}{l}{0}
    \kolicina{l}{k}{0}
    \kolicina{c}{f}{10}
    \kolicina{h}{e}{0}

    \oznaka{s}{0}
    \oznaka{a}{4}
    \oznaka{b}{0}
    \oznaka{c}{12}
    \oznaka{d}{6}
    \oznaka{g}{9}
    \oznaka{j}{24}
    \oznaka{h}{11}
    \oznaka{e}{14}
    \oznaka{f}{17}
    \oznaka{i}{18}
    \oznaka{l}{22}
    \oznaka{k}{19}

    \kandidat{b}{a}[vstopi]
    \kandidat{k}{j}
\end{razvoz}

\begin{razvoz}{grafBs}
    \kolicina{a}{d}{15}
    \kolicina{d}{g}{15}
    \kolicina{g}{j}{10}[izstopi]
    \kolicina{b}{a}{5}
    \kolicina{e}{f}{5}
    \kolicina{f}{i}{10}
    \kolicina{b}{s}{5}
    \kolicina{g}{h}{5}
    \kolicina{i}{l}{0}
    \kolicina{l}{k}{0}
    \kolicina{c}{f}{5}
    \kolicina{h}{e}{5}

    \oznaka{s}{1}
    \oznaka{a}{4}
    \oznaka{b}{1}
    \oznaka{c}{12}
    \oznaka{d}{6}
    \oznaka{g}{9}
    \oznaka{j}{24}
    \oznaka{h}{11}
    \oznaka{e}{14}
    \oznaka{f}{17}
    \oznaka{i}{18}
    \oznaka{l}{22}
    \oznaka{k}{19}

    \kandidat{k}{j}[vstopi]
\end{razvoz}

\clearpage

Optimalna rešitev:

\begin{razvoz}{grafBs}
    \kolicina{a}{d}{15}
    \kolicina{d}{g}{15}
    \kolicina{k}{j}{10}
    \kolicina{b}{a}{5}
    \kolicina{e}{f}{15}
    \kolicina{f}{i}{20}
    \kolicina{b}{s}{5}
    \kolicina{g}{h}{15}
    \kolicina{i}{l}{10}
    \kolicina{l}{k}{10}
    \kolicina{c}{f}{5}
    \kolicina{h}{e}{15}

    \oznaka{s}{1}
    \oznaka{a}{4}
    \oznaka{b}{1}
    \oznaka{c}{12}
    \oznaka{d}{6}
    \oznaka{g}{9}
    \oznaka{j}{21}
    \oznaka{h}{11}
    \oznaka{e}{14}
    \oznaka{f}{17}
    \oznaka{i}{18}
    \oznaka{l}{22}
    \oznaka{k}{19}
\end{razvoz}

Cena razvoza: $5 \cdot 0 + 5 \cdot 3 + 15 \cdot 2 + 15 \cdot 3 + 15 \cdot 2 + 15 \cdot 3 + 15 \cdot 3 + 5 \cdot 5 + 20 \cdot 1 + 10 \cdot 4 + 10 \cdot (-3) + 10 \cdot 2 = 285$

\clearpage
\section*{Naloga 3}

Reši problem razvoza na grafu
z dvofazno simpleksno metodo za omrežje.

\begin{razvoz}[scale=0.9]{grafC}
\end{razvoz}

\subsection*{Rešitev}

Sestavimo omrežje prve faze, da poiščemo začetno dopustno rešitev.

\begin{razvoz}[scale=0.9]{grafCs}
    \kolicina{c}{g}{4}[izstopi]
    \kolicina{d}{g}{2}
    \kolicina{f}{g}{7}
    \kolicina{g}{a}{1}
    \kolicina{g}{b}{5}
    \kolicina{g}{e}{10}

    \oznaka{g}{0}
    \oznaka{c}{-1}
    \oznaka{d}{-1}
    \oznaka{f}{-1}
    \oznaka{a}{1}
    \oznaka{b}{1}
    \oznaka{e}{0}

    \kandidat{c}{b}[vstopi]
    \kandidat{d}{a}
    \kandidat{e}{a}
    \kandidat{f}{b}
    \kandidat{f}{e}
\end{razvoz}

\begin{razvoz}[scale=0.9]{grafCs}
    \kolicina{c}{b}{4}
    \kolicina{d}{g}{2}
    \kolicina{f}{g}{7}
    \kolicina{g}{a}{1}[izstopi]
    \kolicina{g}{b}{1}
    \kolicina{g}{e}{10}

    \oznaka{g}{0}
    \oznaka{c}{1}
    \oznaka{d}{-1}
    \oznaka{f}{-1}
    \oznaka{a}{1}
    \oznaka{b}{1}
    \oznaka{e}{0}

    \kandidat{d}{a}[vstopi]
    \kandidat{e}{a}
    \kandidat{f}{b}
    \kandidat{f}{e}
    \kandidat{g}{c}
\end{razvoz}

\begin{razvoz}[scale=0.9]{grafCs}
    \kolicina{c}{b}{4}
    \kolicina{d}{g}{1}[izstopi]
    \kolicina{f}{g}{7}
    \kolicina{d}{a}{1}
    \kolicina{g}{b}{1}
    \kolicina{g}{e}{10}

    \oznaka{g}{0}
    \oznaka{c}{1}
    \oznaka{d}{-1}
    \oznaka{f}{-1}
    \oznaka{a}{-1}
    \oznaka{b}{1}
    \oznaka{e}{0}

    \kandidat{a}{b}[vstopi]
    \kandidat{f}{b}
    \kandidat{f}{e}
    \kandidat{g}{c}
\end{razvoz}

\begin{razvoz}[scale=0.9]{grafCs}
    \kolicina{c}{b}{4}
    \kolicina{a}{b}{1}
    \kolicina{f}{g}{7}
    \kolicina{d}{a}{2}
    \kolicina{g}{b}{0}[izstopi]
    \kolicina{g}{e}{10}

    \oznaka{g}{0}
    \oznaka{c}{1}
    \oznaka{d}{1}
    \oznaka{f}{-1}
    \oznaka{a}{1}
    \oznaka{b}{1}
    \oznaka{e}{0}

    \kandidat{e}{a}[vstopi]
    \kandidat{e}{d}
    \kandidat{f}{b}
    \kandidat{f}{e}
    \kandidat{g}{c}
    \kandidat{g}{d}
\end{razvoz}

\begin{razvoz}[scale=0.9]{grafCs}
    \kolicina{c}{b}{4}
    \kolicina{a}{b}{1}
    \kolicina{f}{g}{7}
    \kolicina{d}{a}{2}
    \kolicina{e}{a}{0}[izstopi]
    \kolicina{g}{e}{10}

    \oznaka{g}{1}
    \oznaka{c}{1}
    \oznaka{d}{1}
    \oznaka{f}{0}
    \oznaka{a}{1}
    \oznaka{b}{1}
    \oznaka{e}{1}

    \kandidat{f}{b}[vstopi]
    \kandidat{f}{e}
\end{razvoz}

\begin{razvoz}[scale=0.9]{grafCs}
    \kolicina{c}{b}{4}
    \kolicina{a}{b}{1}
    \kolicina{f}{g}{7}[izstopi]
    \kolicina{d}{a}{2}
    \kolicina{f}{b}{0}
    \kolicina{g}{e}{10}

    \oznaka{g}{2}
    \oznaka{c}{1}
    \oznaka{d}{1}
    \oznaka{f}{1}
    \oznaka{a}{1}
    \oznaka{b}{1}
    \oznaka{e}{2}

    \kandidat{f}{e}[vstopi]
\end{razvoz}

\begin{razvoz}[scale=0.9]{grafCs}
    \kolicina{c}{b}{4}
    \kolicina{a}{b}{1}
    \kolicina{f}{e}{7}
    \kolicina{d}{a}{2}
    \kolicina{f}{b}{0}
    \kolicina{g}{e}{3}

    \oznaka{g}{1}
    \oznaka{c}{1}
    \oznaka{d}{1}
    \oznaka{f}{1}
    \oznaka{a}{1}
    \oznaka{b}{1}
    \oznaka{e}{1}
\end{razvoz}

\clearpage

Našli smo optimalno rešitev prve faze, uporabimo dobljeno drevo kot dopustno rešitev originalnega problema:

\begin{razvoz}[scale=0.9]{grafC}
    \kolicina{c}{b}{4}[izstopi]
    \kolicina{a}{b}{1}
    \kolicina{f}{e}{7}
    \kolicina{d}{a}{2}
    \kolicina{f}{b}{0}
    \kolicina{g}{e}{3}

    \oznaka{g}{0}
    \oznaka{c}{-12}
    \oznaka{d}{-7}
    \oznaka{f}{-7}
    \oznaka{a}{-6}
    \oznaka{b}{-3}
    \oznaka{e}{1}

    \kandidat{c}{d}[vstopi]
\end{razvoz}

Optimalna rešitev:

\begin{razvoz}[scale=0.9]{grafC}
    \kolicina{c}{d}{4}
    \kolicina{a}{b}{5}
    \kolicina{f}{e}{7}
    \kolicina{d}{a}{6}
    \kolicina{f}{b}{0}
    \kolicina{g}{e}{3}

    \oznaka{g}{0}
    \oznaka{c}{-11}
    \oznaka{d}{-7}
    \oznaka{f}{-7}
    \oznaka{a}{-6}
    \oznaka{b}{-3}
    \oznaka{e}{1}
\end{razvoz}

Cena razvoza: $4 \cdot 4 + 6 \cdot 1 + 3 \cdot 5 + 3 \cdot 1 + 8 \cdot 7 + 4 \cdot 0 = 96$

\clearpage
\section*{Naloga 4}

Reši problem razvoza na grafu s simpleksno metodo za omrežje
v odvisnosti od parametra $a$.

\begin{razvoz}{grafD}
\end{razvoz}

\end{document}
