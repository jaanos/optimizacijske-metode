\documentclass[14pt]{extarticle}
\usepackage[a4paper, total={18cm, 26cm}]{geometry}
\usepackage[utf8]{inputenc}
\usepackage{amsmath}
\usepackage[oznake]{omrezja}

\tikzstyle{drevo}=[very thick, double, red]
\tikzstyle{zasicena}=[very thick, double]
\tikzstyle{velikost}=[scale=1]

\NewEnviron{razvoz@grafE}{
    \vozlisce{a}[above]{xyz polar cs:angle=90,radius=4}
    \vozlisce{b}[right]{xyz polar cs:angle=18,radius=4}[-2]
    \vozlisce{c}[below]{xyz polar cs:angle=-54,radius=4}[1]
    \vozlisce{d}[below]{xyz polar cs:angle=-126,radius=4}[-3]
    \vozlisce{e}[left]{xyz polar cs:angle=162,radius=4}[4]

    \povezava{b}{a}{2}[sloped]{above}
    \povezava{b}{c}{1}{right}
    \povezava{c}{a}{5}{left}
    \povezava{d}{a}{7}{right}
    \povezava{d}{c}{4}{below}
    \povezava{d}{e}{1}{left}
    \povezava{e}{a}{6}[sloped]{below}
}

\NewEnviron{razvoz@grafEs}{
    \cena{b}{a}{0}
    \cena{b}{c}{0}
    \cena{c}{a}{0}
    \cena{d}{a}{0}
    \cena{d}{c}{0}
    \cena{d}{e}{0}
    \cena{e}{a}{0}

    \begin{razvoz@grafE}
    \end{razvoz@grafE}

    \povezava{a}{c}{1}{right}[bend left=15]
    \povezava{a}{e}{1}[sloped]{above}[bend right=15]
}

\NewEnviron{razvoz@grafF}{
    \tikzstyle{drevo}=[thick]

    \vozlisce{a}[left]{-4,-4}[-2]
    \vozlisce{b}[left]{-4, 4}[]
    \vozlisce{c}[right]{4,-4}[2]
    \vozlisce{d}[right]{4, 4}[]

    \povezava{a}{b}{1}{left}
    \povezava{a}{c}{3}{below}
    \povezava{b}{c}{3}[sloped]{above}
    \povezava{b}{d}{1}{above}
    \povezava{d}{c}{1}{right}
}

\NewEnviron{razvoz@grafG}{
    \vozlisce{a}[above]{-10,  5}[-4]
    \vozlisce{b}[above]{  0,  5}
    \vozlisce{c}[above]{ 10,  5}[8]
    \vozlisce{d}[left]{ -5,  0}
    \vozlisce{e}[225]{  0,  0}[1]
    \vozlisce{f}[right]{  5,  0}
    \vozlisce{g}[below]{-10, -5}[-7]
    \vozlisce{h}[below]{  0, -5}
    \vozlisce{i}[below]{ 10, -5}[2]

    \povezava{a}{b}{3}{above}
    \povezava{a}{d}{2}[sloped]{below}
    \povezava{a}{e}{4}[sloped]{above}
    \povezava{b}{c}{4}{above}
    \povezava{b}{d}{5}[sloped]{above}
    \povezava{b}{e}{2}{right}
    \povezava{c}{f}{3}[sloped]{below}
    \povezava{d}{e}{-2}{above}
    \povezava{d}{g}{1}[sloped]{above}
    \povezava{e}{c}{1}[sloped]{above}
    \povezava{e}{g}{6}[sloped]{below}
    \povezava{e}{i}{3}[sloped]{below}
    \povezava{f}{b}{-1}[sloped]{above}
    \povezava{f}{e}{11}{above}
    \povezava{f}{i}{10}[sloped]{above}
    \povezava{g}{h}{3}{below}
    \povezava{h}{d}{8}[sloped]{below}
    \povezava{h}{e}{5}{right}
    \povezava{h}{f}{3}[sloped]{below}
    \povezava{i}{h}{3}{below}
}

\NewEnviron{razvoz@grafH}{
    \vozlisce{a}[left]{-6, 6}[-9]
    \vozlisce{b}[right]{ 6, 6}[4]
    \vozlisce{c}[right]{ 6,-6}[-8]
    \vozlisce{d}[left]{-6,-6}[-5]
    \vozlisce{e}[above]{-2, 2}[17]
    \vozlisce{f}[below]{ 2,-2}[1]

    \povezava{a}{b}[2]{3}{above}
    \povezava{a}{d}[10]{1}{left}
    \povezava{a}{e}[10]{5}[sloped]{below}
    \povezava{b}{e}[6]{1}[sloped]{above}
    \povezava{c}{b}[7]{1}{right}
    \povezava{c}{f}[8]{1}[sloped]{above}
    \povezava{d}{c}[6]{2}{below}
    \povezava{d}{e}[9]{6}[sloped]{above}
    \povezava{d}{f}[10]{1}[sloped]{below}
    \povezava{f}{b}[8]{4}[sloped]{below}
    \povezava{f}{e}[9]{1}[sloped]{above}
}


\title{Optimizacijske metode -- vaje}
\author{Problem razvoza}
\date{3.4.2020}

\begin{document}

\maketitle

\section*{Naloga 1}

Pokaži, da problem razvoza nima dopustne rešitve.

\begin{razvoz}{grafE}
\end{razvoz}

\clearpage

\subsection*{Rešitev}

Poskusimo poiskati začetno dopustno rešitev.

\begin{razvoz}{grafEs}
    \kolicina{b}{a}{2}
    \kolicina{a}{c}{1}[izstopi]
    \kolicina{d}{a}{3}
    \kolicina{a}{e}{4}

    \oznaka{a}{0}
    \oznaka{b}{0}
    \oznaka{c}{1}
    \oznaka{d}{0}
    \oznaka{e}{1}

    \kandidat{b}{c}[vstopi]
    \kandidat{d}{c}
    \kandidat{d}{e}
\end{razvoz}

\begin{razvoz}{grafEs}
    \kolicina{b}{a}{1}
    \kolicina{b}{c}{1}
    \kolicina{d}{a}{3}[izstopi]
    \kolicina{a}{e}{4}

    \oznaka{a}{0}
    \oznaka{b}{0}
    \oznaka{c}{0}
    \oznaka{d}{0}
    \oznaka{e}{1}

    \kandidat{d}{e}[vstopi]
\end{razvoz}

\begin{razvoz}{grafEs}
    \kolicina{b}{a}{1}
    \kolicina{b}{c}{1}
    \kolicina{d}{e}{3}
    \kolicina{a}{e}{1}

    \oznaka{a}{0}
    \oznaka{b}{0}
    \oznaka{c}{0}
    \oznaka{d}{1}
    \oznaka{e}{1}
\end{razvoz}

Peljemo $1$ po novi povezavi $a \to e$, torej je originalni problem nedopusten.

\clearpage

\section*{Naloga 2}

S pomočjo dualnosti dokaži optimalnost podanega razvoza.

\begin{razvoz}{grafF}
    \tikzstyle{drevo}=[thick]

    \kolicina{a}{b}{x_1 = 1}
    \kolicina{a}{c}{x_2 = 1}
    \kolicina{b}{c}{x_3 = 0}
    \kolicina{b}{d}{x_4 = 1}
    \kolicina{d}{c}{x_5 = 1}

    \oznaka{a}{y_1 = 0}
    \oznaka{b}{y_2 = 1}
    \oznaka{c}{y_3 = 3}
    \oznaka{d}{y_4 = 2}
\end{razvoz}

Linearni program:
\begin{align*}
    \min &\ x_1 + 3 x_2 + 3 x_3 + x_4 + x_5 \\
    -x_1 - x_2 &= -2 \\
    x_1 - x_3 - x_4 &= 0 \\
    x_2 + x_3 + x_5 &= 2 \\
    x_4 - x_5 &= 0 \\
    x_1, x_2, x_3, x_4, x_5 &\ge 0
\end{align*}

Dualni program:
\begin{align*}
    \max -2 y_1 + 2 y_3 \\
    -y_1 + y_2 &\le 1 \\
    -y_1 + y_3 &\le 3 \\
    -y_2 + y_3 &\le 3 \\
    -y_2 + y_4 &\le 1 \\
     y_3 - y_4 &\le 1
\end{align*}

Preverimo dopustnost:
\begin{align*}
    -1 - 1 &= -2 \\
    1 - 0 - 1 &= 0 \\
    1 + 0 + 1 &= 2 \\
    1 - 1 &= 0 \\
    1, 1, 0, 1, 1 &\ge 0
\end{align*}

Postavimo sistem enakosti za neničelne $x_i$:
\begin{align*}
    -y_1 + y_2 &= 1 \\
    -y_1 + y_3 &= 3 \\
    -y_2 + y_4 &= 1 \\
     y_3 - y_4 &= 1
\end{align*}

Rešujemo sistem:
\begin{align*}
    -y_1 + y_2 &= 1 \\
    -y_1 + y_2 &= 1 \\
    -y_2 + y_3 &= 2 \\
     y_3 - y_4 &= 1
\end{align*}

Splošna rešitev:
\begin{alignat*}{2}
    y_3 &&&= y_4 + 1 \\
    y_2 &= y_3 - 2 &&= y_4 - 1 \\
    y_1 &= y_2 - 1 &&= y_4 - 2
\end{alignat*}

Preverimo dopustnost:
$$
3 \ge -y_2 + y_3 = 1 - y_4 + y_4 + 1 = 2
$$

Imamo dopustno rešitev duala (za poljuben $y_4$), torej je dana rešitev originalnega LP optimalna.

Preverimo vrednosti ciljnih funkcij:
\begin{align*}
x_1 + 3x_2 + 3x_3 + x_4 + x_5 &= 6 \\
-2y_1 + 2y_3 = -2y_4 + 4 + 2y_4 + 2 &= 6
\end{align*}

\clearpage

Neoptimalna rešitev:

\begin{razvoz}{grafF}
    \tikzstyle{drevo}=[thick]

    \kolicina{a}{b}{x_1 = 1}
    \kolicina{a}{c}{x_2 = 1}
    \kolicina{b}{c}{x_3 = 1}
    \kolicina{b}{d}{x_4 = 0}
    \kolicina{d}{c}{x_5 = 0}

    \oznaka{a}{y_1 = 0}
    \oznaka{b}{y_2 = 1}
    \oznaka{c}{4 = y_3 = 3 !!}
    \oznaka{d}{y_4 = 2}
\end{razvoz}

Preverimo dopustnost:
\begin{align*}
    -1 - 1 &= -2 \\
    1 - 1 - 0 &= 0 \\
    1 + 1 + 0 &= 2 \\
    0 - 0 &= 0 \\
    1, 1, 1, 0, 0 &\ge 0
\end{align*}

Postavimo sistem enakosti za neničelne $x_i$:
\begin{align*}
    -y_1 + y_2 &= 1 \\
    -y_1 + y_3 &= 3 \\
    -y_2 + y_3 &= 3
\end{align*}

Rešujemo sistem:
\begin{align*}
    -y_1 + y_2 &= 1 \\
    -y_1 + y_2 &= 0 \\
    -y_2 + y_3 &= 3
\end{align*}

Sistem enačb nima rešitve, zato je rešitev originalnega problema neoptimalna!

\clearpage

\section*{Naloga 3}

Poišči najcenejši razvoz na sledečem grafu.

\begin{razvoz}[scale=0.9]{grafG}
    \kolicina{a}{d}{4}
    \kolicina{d}{e}{4}
    \kolicina{e}{c}{3}
    \kolicina{g}{h}{7}
    \kolicina{h}{f}{7}
    \kolicina{f}{i}{2}[izstopi]
    \kolicina{f}{b}{5}
    \kolicina{b}{c}{5}

    \oznaka{a}{0}
    \oznaka{d}{2}
    \oznaka{e}{0}
    \oznaka{c}{1}
    \oznaka{b}{-3}
    \oznaka{f}{-2}
    \oznaka{i}{8}
    \oznaka{h}{-5}
    \oznaka{g}{-8}

    \kandidat{e}{i}[vstopi]
    \kandidat{b}{e}
\end{razvoz}

\begin{razvoz}[scale=0.9]{grafG}
    \kolicina{a}{d}{4}
    \kolicina{d}{e}{4}
    \kolicina{e}{c}{1}
    \kolicina{g}{h}{7}
    \kolicina{h}{f}{7}
    \kolicina{e}{i}{2}
    \kolicina{f}{b}{7}
    \kolicina{b}{c}{7}[izstopi]

    \oznaka{a}{0}
    \oznaka{d}{2}
    \oznaka{e}{0}
    \oznaka{c}{1}
    \oznaka{b}{-3}
    \oznaka{f}{-2}
    \oznaka{i}{3}
    \oznaka{h}{-5}
    \oznaka{g}{-8}

    \kandidat{b}{e}[vstopi]
\end{razvoz}

\begin{razvoz}[scale=0.9]{grafG}
    \kolicina{a}{d}{4}
    \kolicina{d}{e}{4}
    \kolicina{e}{c}{8}
    \kolicina{g}{h}{7}
    \kolicina{h}{f}{7}
    \kolicina{e}{i}{2}
    \kolicina{f}{b}{7}
    \kolicina{b}{e}{7}

    \oznaka{a}{0}
    \oznaka{d}{2}
    \oznaka{e}{0}
    \oznaka{c}{1}
    \oznaka{b}{-2}
    \oznaka{f}{-1}
    \oznaka{i}{3}
    \oznaka{h}{-4}
    \oznaka{g}{-7}

\end{razvoz}

Cena razvoza: $4 \cdot 2 + 4 \cdot (-2) + 2 \cdot 3 + 8 \cdot 1 + 7 \cdot 3 + 7 \cdot 3 + 7 \cdot (-1) + 7 \cdot 2 = 63$

\clearpage

\section*{Naloga 4}

Reši problem razvoza na grafu z omejitvami.

\begin{razvoz}[scale=0.8]{grafH}
    \kolicina{a}{d}{3}
    \kolicina*{d}{c}{6}
    \kolicina{d}{f}{2}
    \kolicina*{c}{f}{8}
    \kolicina{c}{b}{6}[izstopi]
    \kolicina{b}{e}{4}
    \kolicina*{f}{e}{9}
    \kolicina*{a}{b}{2}
    \kolicina{a}{e}{4}

    \oznaka{a}{0}
    \oznaka{d}{1}
    \oznaka{f}{2}
    \oznaka{e}{5}
    \oznaka{b}{4}
    \oznaka{c}{3}

    \kandidat{c}{f}[zasicena,vstopi]
\end{razvoz}

\begin{razvoz}[scale=0.8]{grafH}
    \kolicina{a}{d}{4}
    \kolicina*{d}{c}{6}
    \kolicina{d}{f}{3}
    \kolicina{c}{f}{7}
    \kolicina*{c}{b}{7}
    \kolicina{b}{e}{5}
    \kolicina*{f}{e}{9}
    \kolicina*{a}{b}{2}
    \kolicina{a}{e}{3}

    \oznaka{a}{0}
    \oznaka{d}{1}
    \oznaka{f}{2}
    \oznaka{e}{5}
    \oznaka{b}{4}
    \oznaka{c}{1}

    \kandidat{d}{c}[zasicena, vstopi, izstopi]
\end{razvoz}

\begin{razvoz}[scale=0.8]{grafH}
    \kolicina{a}{d}{4}
    \kolicina{d}{f}{9}
    \kolicina{c}{f}{1}
    \kolicina*{c}{b}{7}
    \kolicina{b}{e}{5}
    \kolicina*{f}{e}{9}
    \kolicina*{a}{b}{2}
    \kolicina{a}{e}{3}

    \oznaka{a}{0}
    \oznaka{d}{1}
    \oznaka{f}{2}
    \oznaka{e}{5}
    \oznaka{b}{4}
    \oznaka{c}{1}
\end{razvoz}

Cena razvoza: $2 \cdot 3 + 3 \cdot 5 + 5 \cdot 1 + 4 \cdot 1 + 9 \cdot 1 + 7 \cdot 1 + 9 \cdot 1 + 1 \cdot 1 = 56$

\end{document}
